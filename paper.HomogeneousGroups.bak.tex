%%%%%%%%%%%%%%%%%%%%%%%% ????? %%%%%%%%%%%%%%%%%%%%%%%%%%%%%%%
\documentclass[12pt]{extarticle} % ????? ?????????
\usepackage[utf8]{inputenc}         % ??????? ?????????
\usepackage[russian]{babel}           % ???????????
\usepackage{amssymb,amsfonts,amsmath,eucal} % ????? AMSLaTeX
\usepackage{graphicx}                 % ??????? ???????
\usepackage{enumerate}                %
\usepackage{color}
\usepackage{hyperref}
\usepackage{anyfontsize}
\usepackage[colorinlistoftodos]{todonotes} % remove

\usepackage[
    left=25mm,
    top=25mm,
    right=45mm,
    bottom=25mm,
    footskip=10mm,
    nohead,
    marginparwidth=40mm,
    driver=xetex
]{geometry}


%%%%%%%%%%%%%%%%%%%%% ????????? ???????? %%%%%%%%%%%%%%%%%%%%%
%%%%%%%%%%%%%%%%%%%%%%%%%%%%%%%%%%%%%%%%%%%%%%%%%%%%%%%%%%%%%%

\title{Однородные группы в универсальных классах абелевых групп}
\author{А.А. Мищенко, В.Н. Ремесленников, А.В. Трейер.}

\newtheorem{theorem}{Теорема}[section]
\newtheorem{lemma}{Лемма}[section]
\newtheorem{proposition}{Предложение}[section]
\newtheorem{statement}{Утверждение}[section]
\newtheorem{corollary}{Следствие}[section]
\newtheorem{definition}{Определение}[section]


\newcommand{\todoi}[1]{\todo[inline]{#1}}
\def\proof{{\noindent{\bf Доказательство.}} }
\def\A{{\mathfrak{A}}}
\def\K{{\mathcal{K}}}
\def\Kec{\mathcal{K}^{ec}}
\def\Kh{\mathcal{K}^{h}}
\def\U{{\mathcal{U}}}
\def\P{{\mathcal{P}}}
\def\F{{\mathcal{F}}}
\def\S{{\mathcal{S}}}
\def\L{{\mathcal{L}}}
\def\C{{\mathcal{C}}}
\def\Z{{\mathbb{Z}}}
\def\N{{\mathbb{N}}}
\def\Q{{\mathbb{Q}}}
\def\Th{{\mathrm{Th}}}
\def\Tha{{\mathrm{Th}_\forall}}
\def\The{{\mathrm{Th}_\exist}}
\def\CG{{\mathrm{CGr}}}
\def\ui{{\mathrm{UI}}}
\def\HP{\textbf{HP}}
\def\JEP{\textbf{JEP}}
\def\AP{\textbf{AP}}
\def\AAP{\textbf{AAP}}


\begin{document}
\maketitle
\tableofcontents
\listoftodos



\section{Введение}

\section{Предварительные сведения}

\section{Канонические группы}


Пусть $\K$ -- главный универсальный класс из $\A_p$, $L = FG(\K)$, $T(L)$ -- множество конечных групп из $L$. Введем характеристику $\delta$ для класса групп $L$ следующим образом:
$$\delta(L) = \left\lbrace 
\begin{array}{l}  
0, \text{ если } L \text{ -- ограниченный класс}; \\ 
1, \text{ в противном случае.}
\end{array} 
\right.$$
Множество $L$ определяет три параметра $a,b \in \N \cup \{0\}$ и $l \in \N \cup \{0, \aleph_0 \}$, где $\aleph_0$ -- счетный кординал, следующим образом:
\begin{enumerate}
\item Параметр $l \in \N \cup \{0, \aleph_0\}$ определим как предельное число с таким свойством: для любого $n \in \N$ $FG(L)$ содержит группу $C = \bigoplus C^l(p^n)$, и существует такое $m$, что $\bigoplus C^{l+1}(p^m) \notin FG(L)$. Если числа $l$ с таким свойством нет среди множества чисел $\N \cup \{0\}$, то определим $l = \aleph_0$.
\item Если $l = \aleph_0$, то $a = b = 0$.
\item Если $l \in \N \cup \{0\}$, тогда числа $a$ и $b$ определяются следующим образом:
\begin{itemize}
 \item Число $a$ -- это максимальное натуральное число со свойством: для любого $m \in \N$ группа $C^m(p^a) \in L$. Если такого числа не существует, то $a = 0$.
 \item Число $b$ -- это минимальное натуральное число со свойством: $\bigoplus C^{l+1}(p^b) \notin FG(L)$.
\end{itemize} 
\end{enumerate}

Нетрудно показать, что $a \leq b$. Для каждого натурального числа $t$ из интервала $(a,b]$ определим число $w_t$ как максимальное число со свойством: группа $C^{w_t}(p^t)$ содержится в классе $L$. Числа $a, b, \{w_t\}$ и $l$ позволяют однозначно по $L$ построить периодическую группу $TC(L)$ таким образом:
\begin{itemize}
\item $TC(L) = C^{\aleph_0}(p^a) \oplus \bigoplus\limits_{a < t \leq b} C^{w_t}(p^t) \oplus C^l(p^\infty)$, если $l \in \N \cup\{0\}$;
\item $TC(L) = C^{\aleph_0}(p^\infty)$, если $l = \aleph_0$.
\end{itemize}

Наконец, $C(L) = TC(L) \oplus B$, где группа $B$ либо равна $0$, если $TC(L)$ неограниченная группа, либо равна $\{0, \Z\}$, если $TC(L)$ ограниченная группа. Обозначим через $C^*(L)$ следующую группу:
\begin{itemize}
\item $C^*(L) = C^{w}(p^a) \oplus \bigoplus\limits_{a < t \leq b} C^{w_t}(p^t) \oplus C^l(p^\infty) \oplus B$, где $w$ -- произвольный ординал, если $l \in \N \cup\{0\}$;
\item $C^*(L) = C^{\aleph_0}(p^\infty)$, если $l = \aleph_0$.
\end{itemize}

Пусть $\K$ -- главный универсальный класс такой, что $FG(\K) = L$. Группу $C(L)$ будем также обозначать через $C(\K)$ и называть \textit{канонической группой} класса $\K$, а группу $C^*(L)$ будем называть \textit{квазиканонической группой} класса $\K$ (будем также обозначать через $C^*(\K)$). Если класс $\K$ -- произвольный универсальный класс, то $L = \bigcup\limits_{i \in I} L_i$, $\K = \bigcup\limits_{i \in I} \K_i$, где $L_i$ -- максимальные направленные множества в $L$, $\K_i = ucl(L_i)$ -- максимальные главные классы в $\K$. Тогда, по определению, $C(L) = C(\K) = \{C(\K_i) | \ i \in I\}$ -- множество канонических групп. Аналогично вводится множество квазиканонических групп $C^*(\K)$.

\begin{theorem}
Пусть $\K$ -- главный универсальный класс и $\K_p = \K \cap \A_p$, $p \in \P$. Тогда верны следующие утверждения:
\begin{enumerate}
\item $T(\K) = \bigoplus\limits_{p \in \P} T_p(\K)$;
\item $\K_p$ -- главный универсальный класс для всех $p \in \P$;
\item Если $\delta(\K) = 0$, то $FG(\K) = FS(\K)$. Если $\delta(\K) = 1$, то $FG(\K) = FS(\K) \oplus \{\Z^k | \ k \in \N\}$;
\item $C(\K) = \bigoplus\limits_{p \in \P} T(C(\K_p)) \oplus B$, где $B = 0$, если $T(C(\K))$ -- неограниченная группа, и $B = \{0, \Z\}$, если $T(C(\K))$ -- ограниченная группа.
\end{enumerate}
\end{theorem}


\section{Свойства $\HP$, $\JEP$, $\AP$ и $\AAP$}
Пусть $\K$ некоторый класс моделей языка $\L$. Будем говорить, что в классе $\K$ выполняются свойства $\HP$, $\JEP$ и $\AP$, если выполнены следующие условия:

\noindent $\HP$ (Hereditary property): Если $A \in \K$ и $B$ конечно порожденная подсистема в $A$, то $B$ изоморфна некоторой системе из $\K$; 

\noindent$\JEP$ (Joint Embeding Property): Если $A, B \in \K$, тогда существует такая система $C \in \K$, что $A$ и $B$ вкладываются в $C$;

\noindent$\AP$ (Amalgamation Property): Если $A, B$ и $C$ системы из $\K$ и $e: A \rightarrow B$ и $f: A \rightarrow C$ вложения, тогда существует система $D \in \K$ и вложения $g : B \rightarrow D$ и $h : C \rightarrow D$ такие, что $ge = hf$.

К сожалению, многие универсальные классы абелевых групп не обладают свойством $\AP$. Поэтому мы ослабляем свойство $\AP$ до $\AAP$.

Пусть $A$ -- абелева группа, $p$-высотой элемента $a \in A$ называется наибольшее натуральное число $k$ такое, что в группе $A$ разрешимо уравнение $p^k x = a$; $p$-высоту элемента $a$ будем обозначать через $h_{p,A}(a)$. 

\noindent$\AAP$ (Admissible Amalgamation Property): Если $A, B$ и $C$ системы из $\K$ и $e: A \rightarrow B$ и $f: A \rightarrow C$ вложения такие, что для любого элемента $a \in A$ $h_{p,B}(e(a)) = h_{p,C}(f(a))$, $p$ -- простое число. Тогда существует система $D \in \K$ и вложения $g : B \rightarrow D$ и $h : C \rightarrow D$ такие, что $ge = hf$.


\begin{theorem}
Пусть $\K$ -- главный универсальный класс абелевых групп, $FG(\K)$ -- множество конечно порожденных подгрупп для $\K$. Тогда множество $FG(\K)$ обладает свойствами $\HP$, $\JEP$ и $\AAP$.
\end{theorem}

\proof Свойство $\HP$ для $FG(\K)$ верно по определению этого множества, в свойство $\JEP$ доказано в теореме~\ref{th:MainClassAp}. 
Докажем, что свойство $\AAP$ (даже более сильное $\AP$) выполнено для $FG(\K)$, разбив доказательство на несколько случаев.

Случай 1: $\K = \A_0$. В этом случае $FG(\K) = \{Z^n | \ n \in \N\}$. Пусть $\varphi: \Z^l \rightarrow \Z^k$, $l \leq k$ -- вложение, и пусть $\varphi(\Z^l)$ -- образ $\varphi$ в $\Z^k$, и $\varphi(Z^l)^{sc}$ -- сервантное замыкание образа в $\Z^k$, и $j$ -- индекс подгруппы $\varphi(\Z^l)$ в сервантном замыкании $\varphi(Z^l)^{sc}$. Пусть $C = \Z^l$, $A = \Z^{k_1}$, $B = \Z^{k_2}$, и пусть вложения $\alpha: C \rightarrow A$ и $\beta: C \rightarrow B$ имеют индексы $j(\alpha)$ и $j(\beta)$ соответственно. Тогда, если $k = \max(k_1, k_2)$, то нетрудно проверить, что вложение $\gamma: A \rightarrow \Z^k$ с индексом $j(\beta)$ и вложение $\delta: B \rightarrow \Z^k$ с индексом $j(\alpha)$ образуют коммутативную диаграмму вложения $C$ в $\Z^k$ так, что $\gamma \alpha = \delta \beta$. Следовательно, в этом случае для $FG(\K)$ выполнено свойство $\AP$, и тем более $\AAP$.

Случай 2: $\K$ -- ограниченный главный класс. В этом случае $FG(\K) = FS(\K)$. Прежде всего определим для двух групп $A$ и $B$ из $FS(\K)$ оператор $\JEP(A,B) = C$, где $C$ -- группа из $FS(\K)$, построенная образом указанным ниже. Для избежания громоздких обозначений будем предполагать, что $\K \subseteq \A_p$, где $p$ -- простое число. По построению $C$ будет подгруппой группы $C(\K)$, где $C(\K)$ -- каноническая группа класса $\K$. По определению, канонические группы для групп $A$ и $B$ есть подгруппы $C(\K)$, и следовательно, есть их общая конечная подгруппа $D$ в $C(\K)$. Построим ее каноническим образом. Если $A = C(p^l)$ и $B = C(p^k)$, $l \geq k$, то $C = C(p^l)$, то есть $\JEP(C(p^l), C(p^k)) = C(p^l)$. Далее пусть $A = C(p^{\alpha_1}) \oplus \ldots \oplus C(p^{\alpha_t}) = A_0 \oplus C(p^{\alpha_t})$ и $B = C(p^{\beta_1}) \oplus \ldots \oplus C(p^{\beta_s}) = B_0 \oplus C(p^{\beta_s})$, где $\alpha_i \leq \alpha_{i+1}$ и $\beta_j \leq \beta_{j+1}$. Тогда полагаем $\JEP(A, B) = \JEP(A_0, B_0) \oplus \JEP(C(p^{\alpha_t}), C(p^{\beta_s}))$, считая, по определению, что $\JEP(A, 0) = A$ и $\JEP(0, B) = B$. Следовательно, $\JEP(A,B) = C$ и проверяется, что $C \in \K$ и по ходу определения строится каноническое вложение $\alpha: A \rightarrow C$ и $\beta: B \rightarrow C$.

Наконец, докажем, что для $FS(\K)$ выполняется свойство $\AAP$. Пусть даны вложения $\alpha: C \rightarrow A$ и $\beta: C \rightarrow B$ такие, что $h_{p,A}(c) = h_{p,B}(c)$ для всех $c \in C$. Это означает, что $A = A_0 \oplus A_1$, $B = B_0 \oplus B_1$, $\alpha(C) \leq A_0$, $\beta(C) \leq B_0$, $A_0 \cong_\psi B_0$. Положим $D = \JEP(A, B)$, и $\gamma, \delta$ -- канонические вложения $A$ в $D$ и $B$ в $D$ соответственно. Причем, ограничения $\gamma$ и $\delta$ на $A_0$ и $B_0$ совпадают $\gamma(a_0) = \delta(\psi(a_0))$. Следовательно, для любого элемента $c \in C$ $\gamma \alpha(c) = \delta \beta(c)$, что и требовалось доказать. Случаи, когда $\K$ -- произвольный ограниченный класс разбираются по аналогии со случаем $\K \subseteq \A_p$.

Случай 3: $\K$ -- неограниченный главный класс. Рассмотрим случай, когда $\K \subseteq A_p$. Если $A = C_1 \oplus Z^{k_1}$ и $B = C_2 \oplus \Z^{k_2}$ , где $C_i$ -- каноническая $p$-группа. Вводим оператор $\JEP(A, B) = \JEP(C_1, C_2) \oplus \Z^k$, где $k = \max(k_1, k_2)$.

Пусть $\alpha: C \rightarrow A$ и $\beta: C \rightarrow B$, $C = C_0 \oplus \Z^l$, $A = A_0 \oplus Z^{k_1}$, $B = B_0 \oplus \Z^{k_2}$, $k = \max(k_1, k_2)$, $l \leq \min(k_1, k_2)$. Индекс вложения $\alpha(\Z^l)$ есть $j(\alpha)$, а индекс вложения $\beta(\Z^l)$ есть $j(\beta)$ (они определены выше). Обозначим через $D_0 = \JEP(A_0, B_0)$ и пусть $D = D_0 \oplus \Z^k$. Тогда существуют вложения $\gamma: A \rightarrow D$ с индексом $\gamma(\Z^{k_1}) = j(\beta)$ и $\delta: B \rightarrow D$ с индексом $\delta(\Z^{k_2}) = j(\alpha)$, такие, что $\gamma\alpha(c) = \delta\beta(c)$ для всех $c \in C$. $\square$

%---------------------------

\section{Предельные группы}

\begin{theorem}[Теорема Фраиссе]
...
\end{theorem}

Будем говорить, что группа $D$ обладает свойством \textit{$h_p$-ультраоднородности}, если любой гомоморфизм сохраняющий высоты между двумя конечными подгруппами $D$ расширяется до автоморфизма группы $D$.

\begin{theorem}[Обобщение теоремы Фраиссе]\label{th:Fraisse}
Пусть универсальный класс абелевых групп $\K$ обладает свойствами $\HP$, $\JEP$ и $\AAP$. Тогда существует единственная с точностью до изоморфизма группа $F_r(\K)$ со свойствами:
\begin{enumerate}
\item $F_r(\K)$ счетная;
\item Любая конечно порожденная группа из $\K$ изоморфна некоторой подгруппе из $F_r(\K)$;
\item $F_r(\K)$ обладает свойством $h_p$-ультраоднородности.
\end{enumerate}
\end{theorem}

\proof \todo{Вставить доказательство.}

В книге \cite{Hodges} группа $F_r(\K)$ называется пределом Фраиссе для класса $FG(\K)$ (отсюда и обозначения ее в этой статье). В других источниках такие пределы называют $P$-ультраоднородными (где $P$ --- фиксированная амальгама из класса систем $\K$).

\begin{corollary}
Если $\K$ универсальный класс для которого существует предел Фраиссе $F$, тогда для класса $\K$ существует и предел $F_r(\K)$, такой, что $F \cong F_r(\K)$.
\end{corollary}

\proof Заметим, что если на классе групп выполнено свойство $\AP$, то будет выполнено и свойство $\AAP$. Если существует предел Фраиссе $F$, то он будет удовлетворят всем свойствам из теоремы \ref{th:Fraisse}, и следовательно, по этой же теореме группы $F$ и $F_r(\K)$ будут изоморфны. $\square$


\section{Структура $\K$-однородных абелевых групп}

\begin{theorem}
Для любого главного универсального класса $\K$ класс экзистенциально замкнутых групп $\Kec$ совпадает с классом $\K$-однородных групп $\Kh$.
\end{theorem}

\todo{Доказать теорему}


% \section{Экзистенциально замкнутые $\K$-группы}

% Пусть $\K$ универсальный класс абелевых групп. Группа $A$ называется \textit{экзистенциально замкнутой} относительно класса $\K$, если любая конечная система уравнений и неравенств над группой $A$ совместная с $\Tha(\K)$ имеет решение в группе $A$. Обозначим $\K_{ec}$ класс экзистенциально замкнутых групп из класса $\K$.

% \begin{lemma}
% Группа $A$ не является экзистенциально замкнутой группой, если выполнено хотя бы одно из следующих условий:
% \begin{enumerate}
% \item $ucl(A) \subset \K$;
% \item $A \big/ T(A)$ не является делимой группой;
% \end{enumerate}
% \end{lemma}

% \proof Рассмотрим универсальное замыкание $ucl(A)$, и построим для него каноническую группу $C = C(ucl(A)).$ Так как $ucl(A) \subset \K$, то существует группа $B \in \K$ такая, что любая конечно порожденная подгруппа группы $B$ принадлежит $ucl(A)$, а сама $B$ не принадлежит $ucl(A)$. Из всех таких групп $B$ выберем минимальную по порядку группу. Рассмотрим подгруппу $B_0$ группы $B$ индекса $p$. Группа $B$ либо имеет вид $B = B_0 \oplus C(p)$, либо получается из группы $B_0$ присоединением корня $B = \langle B_0, b \rangle$. В первом случае
% \todo{Написать уравнение}

% Докажем пункт 2. Так как группа $\overline{A} = A \big/ T(A)$ не является делимой, то выберем элемент $\overline{a} \in  \overline{A}$ и натуральное число $n$ такие, что уравнение $nx = \overline{a}$ не разрешимо в группе $\overline{A}$. Возьмем элемент $a \in A$ являющийся прообразом элемента $\overline{a}$. Тогда уравнение $nx = a$ не разрешимо в группе $A$, так как в противном случае, уравнение $nx = \overline{a}$ было бы разрешимо в группе $\overline{A}$. Из этого следует, что $A$ не является экзистенциально замкнутой группой. \todo{Почему уравнение $nx = a$ разрешимо в классе $\K$?} $\square$ 


% \begin{lemma}
% Пусть $A$ --- абелева группа такая, что $T(A)$ --- неограниченная группа и $A \big/ T(A) \cong \Z^m$ --- группа без кручения. Тогда:
% \begin{enumerate}
% \item $A = T(A) \oplus \Z^m$;
% \item Вложение $i : T(A) \rightarrow A$ является экзистенциально замкнутым.
% \end{enumerate}
% \end{lemma}

% \proof Пункт номер 1 представляет известный факт из теории групп, смотри [???]. \todo{Вставить ссылку на Фукса}

% Для доказательства пункта 2 достаточно доказать, что группа $B = A \big/ T(A)$ (а вместе с ней и группа $A$) дискриминируется группой $T(A)$.

% \todo{Дописать доказательство}

% \begin{corollary}
% Если $A$ экзистенциально замкнутая группа в классе $\K$, тогда:
% \begin{enumerate}
% \item $A = T(A) \oplus B$, где $B$ -- делимая группа;
% \item Если $T(A)$ неограниченная группа, то $T(A)$ также принадлежит классу $\K_{ec}$;
% \item Если $\delta(\K) = 1$ и $T(A)$ ограниченная группа, то $T(A)$ не является экзистенциально замкнутой группой.
% \end{enumerate}
% \end{corollary}


% Пусть $\K$ универсальный класс абелевых групп из $\A_p$. Как и раньше, по примарному универсальному инварианту $\ui_p(\K)$ класса $\K$ можно построить каноническую $p$-групп $C(\K)$, которая имеет вид:
% \begin{equation}\label{eq:CanonnicalGroup}
%  C = C^{\aleph_0}(p^a) \oplus T \oplus C^l(p^\infty) \oplus B,
%  \end{equation}
% где группа $T = \bigoplus\limits_{ a < t \leq b} C^{w_t}(p^t)$, где $w_t = \gamma_{p,t} - \gamma_{p,t+1}$, и группа $B$ либо $\Z$, при $l = 0$ и $\delta(\K) = 1$, либо $B = 0$, в остальных случаях.

% По канонической группе $C(\K)$ построим вспомогательную группу $C^*(\K)$, которая будет иметь вид (\ref{eq:CanonnicalGroup}) только за одним исключением: первое слагаемое $C(p^a)$ будет входить в разложение в степени $w$, где $w$ --- произвольный кардинал.

% Следующая теорема описывает структуру экзистенциально замкнутых групп.
% \begin{theorem}\todo{Проверить формулировку теоремы}
% Пусть $\K$ --- универсальный класс, и пусть $a, b$ и $l$ --- параметры канонической группы $C(\K)$. Тогда экзистенциально замкнутая группа $A$ из класса $\K$ имеет следующую структуру:
% \begin{enumerate}
% \item Если $\K = ucl(T(\K))$ и $\delta(\K) = 1$, тогда
% \begin{itemize}
% \item Если $l = \infty$, то $A$ --- делимая группа из $\A_p$;
% \item Если $l \neq \infty$, то $A = C^*(\K) \oplus (\Q^+)^w.$
% \end{itemize}
% \item Если $T(\K)$ --- ограниченный класс, тогда 
% \begin{itemize}
% \item Если $\delta(\K) = 0$, то $A = C^*(\K);$
% \item Если $\delta(\K) = 1$, то $A = C^*(\K) \oplus (\Q^+)^w.$
% \end{itemize}
% \end{enumerate}
% \end{theorem}

% \proof \todo{Доказать теорему.}

% \todo{Вставить определение модельного компаньона.}
% \begin{theorem}
% Для любого универсального класса $\K$ из $\A_p$, класс $\K_{ec}$ аксиоматизируем, и следовательно, $\Tha(\K)$ имеет модельный компаньон ($\Th(\K_{ec})$).
% \end{theorem}

% \proof \todo{Доказать.}




 
% \section{Простые модели $\Th(\K_{ec})$}

% Пусть $T$ --- непротиворечивая теория языка $L$. Модель $\mathcal{M}$ теории $T$ называется \textit{простой} если для любой модели $\mathcal{N}$ теории $T$ существует элементарное вложение $e : \mathcal{M} \rightarrow \mathcal{N}$. Вложение $e : \mathcal{M} \rightarrow \mathcal{N}$ двух моделей теории $T$ называется \textit{элементарным}, если для любого предложения $\varphi$ теории $T$ с константами из $\mathcal{M}$ верно $$\mathcal{M} \models \varphi \Leftrightarrow \mathcal{N} \models \varphi.$$


% \begin{theorem}
% Для любого универсального класса $\K$ абелевых групп из $\A_p$ для теории $\Th(\K_{ec})$ существует простая модель. Кроме того:
% \begin{enumerate}
% \item Если $\K$ --- ограниченный класс, то $C(\K)$ --- простая модель для теории $\Th(\K_{ec}).$
% \item Если $\K$ --- универсальный класс абелевых групп без кручения, то $\Q^+$ --- простая модель для $\Th(\K_{ec}).$
% \item Если $T_p(\K) \neq 0$ и $\delta(\K) = 1$, то простая модель изоморфна $C(\K) \oplus \Q^+.$
% \end{enumerate}
% \end{theorem}


\bibliographystyle{abbrv}
\bibliography{bibfile}
\end{document}