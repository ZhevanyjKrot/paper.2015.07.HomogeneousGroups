%%%%%%%%%%%%%%%%%%%%%%%% ????? %%%%%%%%%%%%%%%%%%%%%%%%%%%%%%%
\documentclass[12pt]{extarticle} % ????? ?????????
\usepackage[utf8]{inputenc}         % ??????? ?????????
\usepackage[russian]{babel}           % ???????????
\usepackage{amssymb,amsfonts,amsmath,eucal} % ????? AMSLaTeX
\usepackage{graphicx}                 % ??????? ???????
\usepackage{enumerate}                %
\usepackage{color}
\usepackage{hyperref}
\usepackage{anyfontsize}
\usepackage[colorinlistoftodos]{todonotes} % remove

\usepackage[
    left=25mm,
    top=25mm,
    right=45mm,
    bottom=25mm,
    footskip=10mm,
    nohead,
    marginparwidth=40mm,
    % driver=xetex
    driver=pdftex
]{geometry}


%%%%%%%%%%%%%%%%%%%%% ????????? ???????? %%%%%%%%%%%%%%%%%%%%%
%%%%%%%%%%%%%%%%%%%%%%%%%%%%%%%%%%%%%%%%%%%%%%%%%%%%%%%%%%%%%%

\title{Канонические, экзистенциально замкнутые и однородные группы в универсальных классах абелевых групп}
\author{М.Г. Амаглобели, А.А. Мищенко, В.Н. Ремесленников.}

\newtheorem{theorem}{Теорема}[section]
\newtheorem{lemma}{Лемма}[section]
\newtheorem{proposition}{Предложение}[section]
\newtheorem{statement}{Утверждение}[section]
\newtheorem{corollary}{Следствие}[section]
\newtheorem{definition}{Определение}[section]


\newcommand{\todoi}[1]{\todo[inline]{#1}}
\def\proof{{\noindent{\bf Доказательство.}} }
\def\A{{\mathfrak{A}}}
\def\K{{\mathcal{K}}}
\def\Kec{\mathcal{K}^{ec}}
\def\Kh{\mathcal{K}^{h}}
\def\U{{\mathcal{U}}}
\def\P{{\mathcal{P}}}
\def\F{{\mathcal{F}}}
\def\S{{\mathcal{S}}}
\def\L{{\mathcal{L}}}
\def\C{{\mathcal{C}}}
\def\Z{{\mathbb{Z}}}
\def\N{{\mathbb{N}}}
\def\Q{{\mathbb{Q}}}
\def\Th{{\mathrm{Th}}}
\def\Tha{{\mathrm{Th}_\forall}}
\def\The{{\mathrm{Th}_\exist}}
\def\CG{{\mathrm{CGr}}}
\def\ui{{\mathrm{UI}}}
\def\HP{\textbf{HP}}
\def\JEP{\textbf{JEP}}
\def\AP{\textbf{AP}}
\def\AAP{\textbf{AAP}}
\def\Dis{{\mathrm{Dis}}}



\begin{document}
\maketitle




\begin{abstract}
В статье приведены результаты, классифицирующие универсальные классы абелевых групп на языке множеств конечно порожденных групп, замкнутых относительно оператора дискриминируемости. Введено понятие главного универсального класса и канонической группы для него. Для произвольного универсального класса $\K$ описаны группы из класса $\Kec$ -- экзистенциально замкнутых групп относительно универсальной теории класса $\K$. Доказано, что этот класс аксиоматизируем и, следовательно, для универсальной теории $\K$ существует модельный компаньон.
\end{abstract}

% \keywords{абелева группа, универсальный класс, главный универсальный класс, каноническая группа, дискриминируемость классов групп, $K$-экзистенциально замкнутые группы.}


\begin{center}
\begin{minipage}{\textwidth - 2.5cm}
\small
\textbf{Ключевые слова:} абелева группа, универсальный класс, главный универсальный класс, каноническая группа, дискриминируемость классов групп, $K$-экзистенциально замкнутые группы.
\end{minipage}
\end{center}

\tableofcontents

\listoftodos


\section{Введение}

В работах \cite{DMR1, DMR2} по универсальной алгебраической геометрии над произвольными алгебраическими системами доказаны так называемые обобщающие теоремы, в которых, в частности, проблема описания неприводимых алгебраических множеств и неприводимых координатных алгебр для класса алгебр $\K$ языка $L$ прямо увязана с проблемами структурного описания алгебр из универсального замыкания класса $\K$. Это объясняет наше внимание к универсальным классам абелевых групп.

Теории моделей абелевых групп посвящено много работ (они могут быть найдены в статьях, обзорах и книгах \cite{Prest1, Prest2, RR, Hisamiev1,  Palutin1, Palutin2, Ershov, Hodges, Mac, Z}), основой которых служит фундаментальный результат В. Шмелевой \cite{Szm}: две абелевы группы $A$ и $B$ элементарно эквивалентны тогда  и только тогда, когда значения элементарных инвариантов группы $A$ совпадают со значениями элементарных инвариантов группы $B$. 

В рукописи статьи \flqq{}Универсальные инварианты для классов абелевых групп\frqq{} мы ввели универсальные инварианты для универсальных классов абелевых групп и доказали аналоги теоремы Шмелевой для универсальных классов. К сожалению, эти определения довольно громоздкие и у нас нет возможности привести их здесь. В этой статье основными понятиями для характеризации универсальных классов групп будут понятия дискриминируемости одного класса групп другим классом групп и понятие канонической группы для главного универсального класса абелевых групп. Основными результатами данной статьи будут результаты, описывающие  структуру экзистенциально замкнутых групп в универсальном классе (теорема~\ref{th:GroupsInKecAp} и теорема~\ref{th:GroupInKecAny}) и результаты об аксиоматизируемости класса экзистенциально замкнутых групп (теорема~\ref{th:AxiomsOfKec}).

Приведем используемые в статье предварительные сведения и обозначения.

\section{Абелевы группы}
Будем обозначать через $\A$ -- класс всех абелевых групп, через $\A_p$, где $p$ -- простое число, -- класс абелевых групп, периодическая часть которых является $p$-группой, $\A_0$ -- класс, содержащий все абелевы группы без кручения и тривиальную группу, через $FG(\A)$ -- класс конечно порожденных абелевых групп (эти группы являются прямыми суммами циклических групп), через $FS(\A)$ -- множество классов изоморфизмов конечных абелевых групп. Если $\K$ -- произвольный класс абелевых групп, то вводим аналогичные обозначения, заменяя $\A$ на $\K$. Если $A$ -- абелева группа, то через $T(A)$ обозначим периодическую часть группы $A$, а через $T_p(A)$, где $p$ -- простое число, -- максимальную периодическую $p$-подгруппу группы $A$. Класс групп $\K$ будем называть ограниченным, если существует такое натуральное число $m$, что во всех группах из класса $\K$ порядки элементов ограничены числом $m$. Если такого натурального числа $m$ не существует, то класс будем называть неограниченным.

Пусть фиксирована абелева группа $A$. Абелева группа $B$ называется $A$-группой, если зафиксировано вложение $\alpha: A \rightarrow B$. Естественным образом определяется категория абелевых $A$-групп. Пусть $B_1$ и $B_2$ -- группы из этой категории, тогда гомоморфизм $\varphi: B_1 \rightarrow B_2$ называется $A$-гомоморфизмом, если $\varphi(\alpha_1(a)) = \alpha_2(a)$ для всех $a \in A$.

Используемые в статье структурные результаты об абелевых группах можно найти в книгах \cite{Fuchs1, Fuchs2, Ershov}.

\section{Теоретико-модельные обозначения} 
Абелевы группы мы рассматриваем в групповом языке $\L = \{+, -, 0\}$. Если $\K$ -- класс абелевых групп, то через $\Th(\K)$ обозначим его элементарную теорию в языке $\L$, через $\Tha(\K)$ -- универсальную теорию. Обозначим через $A \equiv B$ факт об элементарной эквивалентности групп $A$ и $B$, через $A \equiv_{\forall} B$ факт об универсальной эквивалентности $A$ и $B$. Класс $\K$ называется универсальным классом, если он аксиоматизируем, что можно сделать с помощью универсальных формул языка $\L$. Обозначим через $ucl(\K)$ -- универсальное замыкание класса $\K$. Универсальный класс абелевых групп $\K$ называется главным, если существует такая абелева группа $A$, что $\K = ucl(A)$. Следующие факты хорошо известны в теории моделей.

\begin{proposition}\label{prop:AxiomClass2}
Класс $\K$ универсально аксиоматизируем тогда и только тогда, когда выполнены два следующих условия:
\begin{enumerate}
\item Класс $\K$ аксиоматизируем;
\item Класс $\K$ замкнут относительно операции взятия $\L$-подсистем.
\end{enumerate}
\end{proposition}

\begin{proposition}\label{prop:ClassGeneratesFinetObjects}
Универсальный класс $\K$ $\L$-систем порождается множеством всех своих конечно порожденных систем $FG(\K)$.
\end{proposition}


\section{Свойства $\HP$, $\JEP$, $\AP$ и $\AAP$} 
Пусть $\K$ некоторый класс моделей языка $\L$. Будем говорить, что в классе $\K$ выполняются свойства $\HP$, $\JEP$ и $\AP$, если выполнены следующие условия:

\noindent $\HP$ (Hereditary property): Если $A \in \K$ и $B$ конечно порожденная подсистема в $A$, то $B$ изоморфна некоторой системе из $\K$; 

\noindent$\JEP$ (Joint Embeding Property): Если $A, B \in \K$, тогда существует такая система $C \in \K$, что $A$ и $B$ вкладываются в $C$;

\noindent$\AP$ (Amalgamation Property): Если $A, B$ и $C$ системы из $\K$ и $e: A \rightarrow B$ и $f: A \rightarrow C$ вложения, тогда существует система $D \in \K$ и вложения $g : B \rightarrow D$ и $h : C \rightarrow D$ такие, что $ge = hf$.

К сожалению, многие универсальные классы абелевых групп не обладают свойством $\AP$. Поэтому мы ослабляем свойство $\AP$ до $\AAP$.

Пусть $A$ -- абелева группа, \textit{$p$-высотой} элемента $a \in A$ называется наибольшее натуральное число $k$ такое, что в группе $A$ разрешимо уравнение $p^k x = a$; $p$-высоту элемента $a$ будем обозначать через $h_{p,A}(a)$. 

\noindent$\AAP$ (Admissible Amalgamation Property): Если $A, B$ и $C$ системы из $\K$ и $e: A \rightarrow B$ и $f: A \rightarrow C$ вложения такие, что для любого элемента $a \in A$ $h_{p,B}(e(a)) = h_{p,C}(f(a))$, $p$ -- простое число. Тогда существует система $D \in \K$ и вложения $g : B \rightarrow D$ и $h : C \rightarrow D$ такие, что $ge = hf$.


\begin{theorem}
Пусть $\K$ -- главный универсальный класс абелевых групп, $FG(\K)$ -- множество конечно порожденных подгрупп для $\K$. Тогда множество $FG(\K)$ обладает свойствами $\HP$, $\JEP$ и $\AAP$.
\end{theorem}

\proof Свойство $\HP$ для $FG(\K)$ верно по определению этого множества, в свойство $\JEP$ доказано в теореме~\ref{th:MainClassAp}. 
Докажем, что свойство $\AAP$ (даже более сильное $\AP$) выполнено для $FG(\K)$, разбив доказательство на несколько случаев.

Случай 1: $\K = \A_0$. В этом случае $FG(\K) = \{Z^n | \ n \in \N\}$. Пусть $\varphi: \Z^l \rightarrow \Z^k$, $l \leq k$ -- вложение, и пусть $\varphi(\Z^l)$ -- образ $\varphi$ в $\Z^k$, и $\varphi(Z^l)^{sc}$ -- сервантное замыкание образа в $\Z^k$, и $j$ -- индекс подгруппы $\varphi(\Z^l)$ в сервантном замыкании $\varphi(Z^l)^{sc}$. Пусть $C = \Z^l$, $A = \Z^{k_1}$, $B = \Z^{k_2}$, и пусть вложения $\alpha: C \rightarrow A$ и $\beta: C \rightarrow B$ имеют индексы $j(\alpha)$ и $j(\beta)$ соответственно. Тогда, если $k = \max(k_1, k_2)$, то нетрудно проверить, что вложение $\gamma: A \rightarrow \Z^k$ с индексом $j(\beta)$ и вложение $\delta: B \rightarrow \Z^k$ с индексом $j(\alpha)$ образуют коммутативную диаграмму вложения $C$ в $\Z^k$ так, что $\gamma \alpha = \delta \beta$. Следовательно, в этом случае для $FG(\K)$ выполнено свойство $\AP$, и тем более $\AAP$.

Случай 2: $\K$ -- ограниченный главный класс. В этом случае $FG(\K) = FS(\K)$. Прежде всего определим для двух групп $A$ и $B$ из $FS(\K)$ оператор $\JEP(A,B) = C$, где $C$ -- группа из $FS(\K)$, построенная образом указанным ниже. Для избежания громоздких обозначений будем предполагать, что $\K \subseteq \A_p$, где $p$ -- простое число. По построению $C$ будет подгруппой группы $C(\K)$, где $C(\K)$ -- каноническая группа класса $\K$. По определению, канонические группы для групп $A$ и $B$ есть подгруппы $C(\K)$, и следовательно, есть их общая конечная подгруппа $D$ в $C(\K)$. Построим ее каноническим образом. Если $A = C(p^l)$ и $B = C(p^k)$, $l \geq k$, то $C = C(p^l)$, то есть $\JEP(C(p^l), C(p^k)) = C(p^l)$. Далее пусть $A = C(p^{\alpha_1}) \oplus \ldots \oplus C(p^{\alpha_t}) = A_0 \oplus C(p^{\alpha_t})$ и $B = C(p^{\beta_1}) \oplus \ldots \oplus C(p^{\beta_s}) = B_0 \oplus C(p^{\beta_s})$, где $\alpha_i \leq \alpha_{i+1}$ и $\beta_j \leq \beta_{j+1}$. Тогда полагаем $\JEP(A, B) = \JEP(A_0, B_0) \oplus \JEP(C(p^{\alpha_t}), C(p^{\beta_s}))$, считая, по определению, что $\JEP(A, 0) = A$ и $\JEP(0, B) = B$. Следовательно, $\JEP(A,B) = C$ и проверяется, что $C \in \K$ и по ходу определения строится каноническое вложение $\alpha: A \rightarrow C$ и $\beta: B \rightarrow C$.

Наконец, докажем, что для $FS(\K)$ выполняется свойство $\AAP$. Пусть даны вложения $\alpha: C \rightarrow A$ и $\beta: C \rightarrow B$ такие, что $h_{p,A}(c) = h_{p,B}(c)$ для всех $c \in C$. Это означает, что $A = A_0 \oplus A_1$, $B = B_0 \oplus B_1$, $\alpha(C) \leq A_0$, $\beta(C) \leq B_0$, $A_0 \cong_\psi B_0$. Положим $D = \JEP(A, B)$, и $\gamma, \delta$ -- канонические вложения $A$ в $D$ и $B$ в $D$ соответственно. Причем, ограничения $\gamma$ и $\delta$ на $A_0$ и $B_0$ совпадают $\gamma(a_0) = \delta(\psi(a_0))$. Следовательно, для любого элемента $c \in C$ $\gamma \alpha(c) = \delta \beta(c)$, что и требовалось доказать. Случаи, когда $\K$ -- произвольный ограниченный класс разбираются по аналогии со случаем $\K \subseteq \A_p$.

Случай 3: $\K$ -- неограниченный главный класс. Рассмотрим случай, когда $\K \subseteq A_p$. Если $A = C_1 \oplus Z^{k_1}$ и $B = C_2 \oplus \Z^{k_2}$ , где $C_i$ -- каноническая $p$-группа. Вводим оператор $\JEP(A, B) = \JEP(C_1, C_2) \oplus \Z^k$, где $k = \max(k_1, k_2)$.

Пусть $\alpha: C \rightarrow A$ и $\beta: C \rightarrow B$, $C = C_0 \oplus \Z^l$, $A = A_0 \oplus Z^{k_1}$, $B = B_0 \oplus \Z^{k_2}$, $k = \max(k_1, k_2)$, $l \leq \min(k_1, k_2)$. Индекс вложения $\alpha(\Z^l)$ есть $j(\alpha)$, а индекс вложения $\beta(\Z^l)$ есть $j(\beta)$ (они определены выше). Обозначим через $D_0 = \JEP(A_0, B_0)$ и пусть $D = D_0 \oplus \Z^k$. Тогда существуют вложения $\gamma: A \rightarrow D$ с индексом $\gamma(\Z^{k_1}) = j(\beta)$ и $\delta: B \rightarrow D$ с индексом $\delta(\Z^{k_2}) = j(\alpha)$, такие, что $\gamma\alpha(c) = \delta\beta(c)$ для всех $c \in C$. $\square$



Будем говорить, что группа $D$ обладает свойством \textit{$h_p$-ультраоднородности}, если любой гомоморфизм сохраняющий высоты между двумя конечными подгруппами $D$ расширяется до автоморфизма группы $D$.

\begin{theorem}[Обобщение теоремы Фраиссе]\label{th:Fraisse}
Пусть главный универсальный класс абелевых групп $\K$ такой, что $FG(\K)$ обладает свойствами $\HP$, $\JEP$ и $\AAP$. Тогда существует единственная с точностью до изоморфизма группа $F_r(\K)$ со свойствами:
\begin{enumerate}
\item $F_r(\K)$ счетная;
\item Любая конечно порожденная группа из $\K$ изоморфна некоторой подгруппе из $F_r(\K)$ ($FG(\K) = FG(F_r(\K))$);
\item $F_r(\K)$ обладает свойством $h_p$-ультраоднородности.
\end{enumerate}
\end{theorem}

Перед доказательством теоремы определим вспомогательное понятие и докажем лемму. Систему $D$ будем называть \textit{слабо $h_p$-однородной} если для любых двух конечно порожденных подструктур $A$ и $B$ таких, что $A \subseteq B$ и $f: A \rightarrow D$ -- вложение, сохраняющее высоты, существует вложение $g: B \rightarrow D$, сохраняющее высоты, которое продолжает вложение $f$. Заметим, что если система является $h_p$-ультраоднородной, то она и является слабо $h_p$-однородной.

\begin{lemma}\label{lemma:WeaklyHomogeneouseEmbeding}
Пусть $C$ и $D$ две счетные системы. Если $FG(C) \subseteq FG(D)$ и $D$ -- слабо $h_p$-однородная система, тогда $C$ вкладывается в $D$ и любое вложение конечно порожденной подсистемы $C$ может быть продолжено до вложения $C$ в $D$.
\end{lemma}

\proof Пусть $f_0: A_0 \rightarrow D$ -- вложение, сохраняющее высоты, конечно порожденной подсистемы $A_0$ системы $C$ в систему $D$. Мы продолжим $f_0$ до вложения $f_\omega: C \rightarrow D$. Так как $C$ -- счетная система, то она может быть представлена как объединение $\cup_{n < \omega} A_n$ цепи конечно порожденных подсистем, начинающихся с $A_0$. По индукции определим цепочку продолжающихся вложений $f_n: A_n \rightarrow D$, сохраняющих высоты. Первое вложение $f_0$ уже задано. Предположим мы определили вложение $f_n$. Так как $FG(C) \subseteq FG(D)$, то существует изоморфизм $g: A_{n+1} \rightarrow B$, где $B$ -- подсистема в $D$ (так как $g$ изоморфизм, то он сохраняет высоты). Тогда отображение $f_n \cdot g^{-1}$ вкладывает $g(A_n)$ в $D$, и согласно условию слабой $h_p$-однородности это вложение продолжается до вложения $h: B \rightarrow D$, сохраняющего высоты. Определим $f_{n+1}: A_{n+1} \rightarrow D$ как $hg$. Заметим, что $f_n \subseteq f_{n+1}$, и что вложение $f_{n+1}$ сохраняет высоты. Таким образом, мы построили цепочку отображений $f_n$. Определим $f_\omega$ как объединение цепочки вложений $f_n$ $(n < \omega)$. $\square$

\medskip
\noindent{\bf Доказательство теоремы \ref{th:Fraisse}.} Докажем единственность. Пусть $C$ и $D$ две $h_p$-ультраоднородные системы для которых выполнено $FG(C) = FG(D)$. Тогда, так как они будут и слабо $h_p$-однородными, то по лемме \ref{lemma:WeaklyHomogeneouseEmbeding} они будут изоморфны.

Докажем существование. Построим цепь $(D_i : i < \omega)$ систем из $FG(\K)$ такую, что будет выполнено следующее свойство: если $A, B \in \K$, $A \subseteq B$ и существует вложение $f: A \rightarrow D_i$ для некоторого $i$, которое сохраняет высоты, тогда существует $j > i$ и вложение $g : B \rightarrow D_j$, сохраняющее высоты, которое продолжает вложение $f$.

Обозначим через $\P$ счетное множество пар $(A, B)$ систем из $\K$ таких, что $A \subseteq B$. Мы можем выбрать множество $\P$ так, чтобы оно содержало представителя каждого типа изоморфизма для каждой пары. Рассмотрим взаимно однозначное отображение $\pi: \omega \times \omega \rightarrow \omega$ такое, что $\pi(i, j) \geq i$ для всех $i$ и $j$. Пусть $D_0$ -- произвольная система из $\K$. Построим цепь $(D_i : i < \omega)$ по индукции. Предположим система $D_k$ выбрана. Составим множество троек $\{(f_{kj}, A_{kj}, B_{kj}) |\ (A_{kj}, B_{kj}) \in \P \text{ и } f_{kj}: A_{kj} \rightarrow D_k \}$. Если $k = \pi(i,j)$ тогда строим $D_{k+1}$ по свойству $\AAP$ так, что вложение $f_{ij}$ продолжается до вложения $B_{ij}$ в $D_{k+1}$.


Определим структуру $D$ как объединение $\cup_{i < \omega} D_i$. Нетрудно показать, что $FG(D) \subseteq FG(\K)$. Покажем обратное включение. Рассмотрим $A \in FG(\K)$, тогда по свойству $\JEP$ для системы $A$ и $D_0$ существует такая система $B \in FG(\K)$, что $A \subseteq B$ и $D_0$ вкладываются в $B$. По свойству построения цепи $(D_i)$ существует вложение $B$ в некоторую структуру $D_j$. Следовательно, $A$ и $B$ лежат в $FG(D)$. По свойству построения цепи структура $D$ является слабо $h_p$-однородной. Применяя лемму \ref{lemma:WeaklyHomogeneouseEmbeding} при условии, что $C = D$ и $D$ слабо $h_p$-однородная можно показать, что $D$ будет $h_p$-ультраоднородной. Действительно, так как в данном случае любое вложение $f: A \rightarrow D$ (где $A$ -- конечная подсистема $D$) может быть расширено до вложения $D$ в $D$. $\square$  



\section{Оператор дискриминируемости} 
Пусть $L \subseteq FG(\A)$ -- произвольное множество конечно порожденных абелевых групп. Какие условия должны быть выполнены для $L$, чтобы существовал универсальный класс $\K$ такой, что $FG(\K) = L$? Ответ на этот вопрос дает теорема~\ref{th:LDisL}, доказываемая с помощью обобщающих теорем из \cite{DMR1, DMR2}. Приведем необходимые определения

\begin{definition}
Группа $A$ дискриминируется группой $B$ тогда и только тогда, когда для любого набора нетривиальных элементов $a_1, \ldots, a_n \in A$ существует гомоморфизм $\varphi : A \rightarrow B$ такой, что образы элементов $\varphi(a_1), \ldots, \varphi(a_n)$ не равны 0.
\end{definition}

\begin{definition}
Пусть $L$ некоторое множество конечно порожденных абелевых групп. Обозначим через $\Dis(L)$ -- множество всех таких конечно порожденных групп, которые дискриминируются группами из множества $L$.
\end{definition}

\begin{definition}
Для абелевой группы $A$ определим класс $A-\Dis(A)$ как класс всех таких абелевых $A$-групп $B$, что для любого набора нетривиальных элементов $b_1, \ldots, b_n \in B$ существует $A$-гомоморфизм $\varphi: B \rightarrow A$ такой, что образы элементов $\varphi(b_1), \ldots, \varphi(b_n)$ не равны 0.
\end{definition}


\begin{theorem}\label{th:LDisL}
В обозначениях выше $L = FG(\K)$, где $\K$ -- некоторый универсальный класс, тогда и только тогда, когда для $L$ выполнено равенство $$L = \Dis(L).$$
\end{theorem}


\section{Частично упорядоченные множества} 
Пусть $\langle L, \preceq \rangle$ -- частично упорядоченное множество. Множество $L$ называется направленным, если для любых двух элементов $x$ и $y$ из $L$ существует элемент $z \in L$ такой, что $x \preceq z$ и $y \preceq z$. С помощью леммы Цорна доказывается следующий результат.
\begin{proposition}\label{prop:LIsUnion}
Любое частично упорядоченное множество $L$ представимо в виде объединения $L = \bigcup\limits_{i \in I} L_i$, где $L_i$ -- максимальное направленное подмножество в $L$, $I$ -- есть множество индексов всех таких подмножеств в $L$.
\end{proposition}


\section{Характеризация универсальных классов с помощью множеств конечно порожденных групп} 
В следующих двух теоремах мы даем характеризацию универсальных классов абелевых групп на языке свойств множества $FG(\K)$.


\begin{theorem}\label{th:MainClassAp}
Пусть $L \subseteq FG(\A)$ -- подмножество в множестве всех конечно порожденных абелевых групп. Тогда следующие условия для $L$ эквивалентны:
\begin{enumerate}
\item $\langle L, \preceq \rangle$ -- направленное множество и $L = \Dis(L)$;
\item для $L$ выполнены свойства $\HP$, $\JEP$ и, кроме того, если $C$ -- конечная группа из $L$, то $C \in L$ тогда и только тогда, когда $T_p(C) \in L$ для всех простых чисел $p$; и если, кроме того, $\Z \in L$, то $C \oplus \Z^k \in L$, $k \in \N$.
\item Существует абелева группа $A$ такая, что $FG(ucl(A)) = L$.
\end{enumerate}
\end{theorem}

\begin{corollary}
Пусть $L_1$ и $L_2$ -- два подмножества конечно порожденных абелевых групп и $L_i = \Dis(L_i)$, $\K_i = ucl(L_i)$ для $i = 1,2$. Тогда $\K_1 = \K_2$ тогда и только тогда, когда $L_1 = L_2$.
\end{corollary}

Если $K$ -- произвольный универсальный класс, то по предложению~\ref{prop:LIsUnion} соответствующее частично упорядоченное множество $FG(\K)$ раскладывается в объединение максимальных направленных множеств.

\begin{theorem}
В вышеприведенных обозначениях пусть $L = \Dis(L)$, $L = \bigcup\limits_{i \in I} L_i$ и $\K = ucl(L)$. Тогда верны следующие утверждения:
\begin{enumerate}
\item $L_i = \Dis(L_i)$, для всех $i \in I$;
\item если $\K_i = ucl(L_i)$ -- соответствующий главный класс, то $\K = \bigcup\limits_{i \in I} \K_i$.
\end{enumerate}
\end{theorem}


\section{Канонические группы} 
Пусть $\K$ -- главный универсальный класс из $\A_p$, $L = FG(\K)$, $T(L)$ -- множество конечных групп из $L$. Введем характеристику $\delta$ для класса групп $L$ следующим образом:
$$\delta(L) = \left\lbrace 
\begin{array}{l}  
0, \text{ если } L \text{ -- ограниченный класс}; \\ 
1, \text{ в противном случае.}
\end{array} 
\right.$$
Множество $L$ определяет три параметра $a,b \in \N \cup \{0\}$ и $l \in \N \cup \{0, \aleph_0 \}$, где $\aleph_0$ -- счетный кординал, следующим образом:
\begin{enumerate}
\item Параметр $l \in \N \cup \{0, \aleph_0\}$ определим как предельное число с таким свойством: для любого $n \in \N$ $FG(L)$ содержит группу $C = \bigoplus C^l(p^n)$, и существует такое $m$, что $\bigoplus C^{l+1}(p^m) \notin FG(L)$. Если числа $l$ с таким свойством нет среди множества чисел $\N \cup \{0\}$, то определим $l = \aleph_0$.
\item Если $l = \aleph_0$, то $a = b = 0$.
\item Если $l \in \N \cup \{0\}$, тогда числа $a$ и $b$ определяются следующим образом:
\begin{itemize}
 \item Число $a$ -- это максимальное натуральное число со свойством: для любого $m \in \N$ группа $C^m(p^a) \in L$. Если такого числа не существует, то $a = 0$.
 \item Число $b$ -- это минимальное натуральное число со свойством: $\bigoplus C^{l+1}(p^b) \notin FG(L)$.
\end{itemize} 
\end{enumerate}

Нетрудно показать, что $a \leq b$. Для каждого натурального числа $t$ из интервала $(a,b]$ определим число $w_t$ как максимальное число со свойством: группа $C^{w_t}(p^t)$ содержится в классе $L$. Числа $a, b, \{w_t\}$ и $l$ позволяют однозначно по $L$ построить периодическую группу $TC(L)$ таким образом:
\begin{itemize}
\item $TC(L) = C^{\aleph_0}(p^a) \oplus \bigoplus\limits_{a < t \leq b} C^{w_t}(p^t) \oplus C^l(p^\infty)$, если $l \in \N \cup\{0\}$;
\item $TC(L) = C^{\aleph_0}(p^\infty)$, если $l = \aleph_0$.
\end{itemize}

Наконец, $C(L) = TC(L) \oplus B$, где группа $B$ либо равна $0$, если $TC(L)$ неограниченная группа, либо равна $\{0, \Z\}$, если $TC(L)$ ограниченная группа. Обозначим через $C^*(L)$ следующую группу:
\begin{itemize}
\item $C^*(L) = C^{w}(p^a) \oplus \bigoplus\limits_{a < t \leq b} C^{w_t}(p^t) \oplus C^l(p^\infty) \oplus B$, где $w$ -- произвольный ординал, если $l \in \N \cup\{0\}$;
\item $C^*(L) = C^{\aleph_0}(p^\infty)$, если $l = \aleph_0$.
\end{itemize}

Пусть $\K$ -- главный универсальный класс такой, что $FG(\K) = L$. Группу $C(L)$ будем также обозначать через $C(\K)$ и называть \textit{канонической группой} класса $\K$, а группу $C^*(L)$ будем называть \textit{квазиканонической группой} класса $\K$ (будем также обозначать через $C^*(\K)$). Если класс $\K$ -- произвольный универсальный класс, то $L = \bigcup\limits_{i \in I} L_i$, $\K = \bigcup\limits_{i \in I} \K_i$, где $L_i$ -- максимальные направленные множества в $L$, $\K_i = ucl(L_i)$ -- максимальные главные классы в $\K$. Тогда, по определению, $C(L) = C(\K) = \{C(\K_i) | \ i \in I\}$ -- множество канонических групп. Аналогично вводится множество квазиканонических групп $C^*(\K)$.

\begin{theorem}
Пусть $\K$ -- главный универсальный класс и $\K_p = \K \cap \A_p$, $p \in \P$. Тогда верны следующие утверждения:
\begin{enumerate}
\item $T(\K) = \bigoplus\limits_{p \in \P} T_p(\K)$;
\item $\K_p$ -- главный универсальный класс для всех $p \in \P$;
\item Если $\delta(\K) = 0$, то $FG(\K) = FS(\K)$. Если $\delta(\K) = 1$, то $FG(\K) = FS(\K) \oplus \{\Z^k | \ k \in \N\}$;
\item $C(\K) = \bigoplus\limits_{p \in \P} T(C(\K_p)) \oplus B$, где $B = 0$, если $T(C(\K))$ -- неограниченная группа, и $B = \{0, \Z\}$, если $T(C(\K))$ -- ограниченная группа.
\end{enumerate}
\end{theorem}

\section{$\K$-экзистенциально замкнутые группы} 
Пусть $\K$ -- универсальный класс. Группа $A$ называется $\K$-экзистенциально замкнутой, если любая конечная система уравнений и неравенств над группой $A$ совместная с $\Tha(\K)$ имеет решение в $A$. Обозначим через $\Kec$ -- класс экзистенциально замкнутых групп в классе $\K$.

\todo{Возможно следующее предложение можно убрать, оно дублируется следующим}
\begin{proposition}
Если $A$ -- экзистенциально замкнутая группа в главном универсальном классе $\K$, то:
\begin{enumerate}
\item $A = T(A) \oplus B$, где $B$ -- делимая группа без кручения, либо нулевая подгруппа;
\item Если $T(A)$ -- неограниченная группа, то $T(A)$ также принадлежит $\Kec$;
\item Если $\delta(\K) = 1$ и $T(A)$ -- ограниченная группа, то $T(A)$ не является экзистенциально замкнутой группой.
\end{enumerate}
\end{proposition}

\begin{proposition}
Пусть $\K$ -- главный универсальный класс, $L = FG(\K)$, $A$ -- группа из $\Kec$. Тогда:
\begin{enumerate}
\item Если $D = A(x_1, \ldots, x_n)$ -- конечное расширение группы $A$ с помощью конечной системы элементов и $D \in \K$, то ${D \in A-\Dis(A)}$;
\item Если $T(A)$ -- периодическая часть группы $A$, то
\begin{equation}\label{eq:EC-group}
A = T(A) \oplus B,
\end{equation}
где $B$ -- делимая группа без кручения;
\item Если $\delta(T(A)) = 1$, то $T(A)$ также принадлежит $\Kec$;
\item Если $\delta(T(A)) = 0$ и $\delta(\K) = 0$, то $B = 0$ в разложении (\ref{eq:EC-group});
\item Если $\delta(T(A)) = 0$ и $\delta(\K) = 1$, то $B \neq 0$ в разложении (\ref{eq:EC-group});
\end{enumerate}
\end{proposition}

\proof \begin{enumerate}
\item Если для $A$ выполнено условие, то $D \in \Kec$ по определению $\Kec$-группы. Верно и обратное утверждение. Поэтому пункт~1 -- это одно из определений $\K$-экзистенциально замкнутой системы.

\item Если $A \Big/ T(A)$ не является делимой группой, то существует элемент $\overline{a} \in T(A)$ и простое число $p$ такие, что уравнение $px = a$ не имеет решение по модулю $T(A)$ и, следовательно, в $A$. Пусть $x$ -- решение данного уравнения, $B = \langle A, x\rangle$ -- абелева группа. Тогда $FG(A) = FG(B)$, и группа $B \in \K$. Следовательно, указанное уравнение совместно с $\Tha(\K)$, но не имеет решения в $A$, что не так. Отсюда следует, что $A \Big/ T(A)$ -- делимая группа, и $A = T(A) \oplus B$, где $B \cong A \Big/ T(A)$.

\item Если $\delta(T(A)) = 1$ и $D = T(A)(x_1, \ldots, x_n)$ -- конечное расширение $T(A)$ в $A$. Тогда $D = T(A) \oplus \Z^k$ и по лемме~??? $D$ $T(A)$-дискриминируется $T(A)$. Отсюда следует результат. \todo{Здесь нужна лемма о башне расширений.}

\item Очевидное утверждение.

\item Так как $\Z$ не дискриминируется $T(A)$ в этом случае, то $B \neq 0$. $\square$
\end{enumerate}






Для группы $A$ через $P(A)$ обозначим три параметра $a, b$ и $l$, вычисленных для класса $ucl(A)$. Для двух групп $A_1$ и $A_2$ будем писать: $P(A_1) = P(A_2)$, если $a_1 = a_2, b_1 = b_2$ и $l_1 = l_2$; $P(A_1) > P(A_2)$, если $a_1 \geq a_2, b_1 \geq b_2, l_1 \geq l_2$ и хотя бы одно из неравенств строгое.

\begin{lemma}\label{lemma:AinOrnotinKec}
Пусть $\K$ -- главный универсальный класс и пусть $A \in \K$ и $D = A(x_1, \ldots, x_n) \in \K$. Тогда:
\begin{enumerate}
\item Если $A$ не является сервантной подгруппой группы $D$ в некоторой $D = A(x_1, \ldots, x_n)$ из $\K$, то $A$ не является экзистенциально замкнутой группой в $\K$;

\item Если группа $A$ является сервантной подгруппой в $D = A \oplus C \oplus \Z^k \in \K$, где $C$ -- конечная группа, $k \in \N$, но если $P(D) > P(A)$, то $A \notin \Kec$;

\item Пусть группа $A$ для любой группы $D = A(x_1, \ldots, x_n) \in \K$ обладает свойствами:
\begin{itemize}
\item $A$ сервантна в $D$;
\item $P(A) = P(D)$;
\end{itemize}
то группа $A$ является экзистенциальной группой, то есть $A \in \Kec$.
\end{enumerate}
\end{lemma}

\proof Докажем пункт 1 теоремы. Если существует элемент $a \neq 0$ из $A$ такой, что уравнение $nx = a$ не имеет решения в $D$, то ясно, что $D \notin A-\Dis(A)$.

Докажем пункт 2. Если $P(D) > P(A)$, то и $ucl(D) \supset ucl(A)$ и, следовательно, $A \notin \Kec$\todo{добавить ссылку}.

Докажем пункт 3. Утверждение следует из определения, ибо любое вложение $A \rightarrow E$, $E \in \K$ является $\exists_1$-вложением. $\square$



\begin{theorem}\label{th:GroupsInKecAp}
Пусть $\K$ -- главный универсальный класс из $\A_p$. Тогда:
\begin{enumerate}
\item Если характеристика $l(\K) = \aleph_0$, то $\K = \A_p$ и $\Kec$ состоит из делимых групп вида $A = T(A) \oplus B$, где $T(A) = C^{\chi_1}(p^\infty)$, $B = (\Q^+)^{\chi_2}$, $\chi_1, \chi_2$ -- кардиналы;

\item Если $l(\K) = l$, $\delta(T(\K)) = 1$, где $l$ -- натуральное число, и $A \in \Kec$, то $A = T(A) \oplus B$, где $B$ -- делимая группа без кручения и $T(A)$ также принадлежит $\Kec$ и $T(A) \in C^*(\K)$, где $C^*(\K)$ -- есть множество квазиканонических групп для $\K$;

\item Если $\delta(T(\K)) = 0$, то 
\begin{itemize}
\item Если $\delta(\K) = 0$, то $\Kec = C^*(\K)$;
\item Если $\delta(\K) = 1$, то $\Kec$ состоит из групп вида $A = A_0 \oplus B$, где $A_0 \in C^*(T(\K))$, $B \neq 0$ -- делимая группа без кручения.
\end{itemize}
\end{enumerate}
\end{theorem}

\proof Докажем пункт 1. Если $A$ не является делимой группой, то существует элемент $a \neq 0$ из $A$ и простое число $q$ такие, что уравнение $qx = a$ неразрешимо в $A$. Тогда группа $B = \langle A, x\rangle$ принадлежит $\K$, но не $A$-дискриминируется группой $A$, что не так (см. пункт~1).

Докажем пункт 2. Пусть $A = C(p^a) \oplus T \oplus C^l(p^\infty)$ -- группа из множества $C^*(\K)$ и $D = A(x_1, \ldots, x_n)$ -- конечное расширение $A$ в $\K$. $D = T(D) \oplus \Z^k$, $k \leq n$, $A \subseteq T(D)$. Так как $\K = ucl(A) = ucl(T(D))$, то $FS(A) = FS(T(D))$ и отсюда следует, что $T(D)$ -- $A$-дискриминируется $A$ и $D$ -- $T(D)$-дискриминируется $T(D)$, а отсюда в силу транзитивности $D$ -- $A$-дискриминируется $A$. Следовательно, $A \in \Kec$ по пункту~1. Аналогично доказывается, что если $A = T(A) \oplus B$, где $T(A) \in C^*(\K)$, а $B$ -- делимая группа без кручения, то $B \in \Kec$.

Обратно. Докажем, что если $A \in \Kec$, то она имеет вид как указано в формулировке теоремы. В силу .... достаточно доказать это утверждение для $A = T(A)$. Итак, пусть $A \notin C^*(\K)$. Тогда, если $ucl(A) \neq \K$, то $FS(A) \subset FS(\K)$ и, следовательно, существует конечная группа $C \in \K$, которой нет в $A$. Понятно, что $C$ не $A$-дискриминируется $A$, и следовательно, $A \notin \Kec$.
 
Итак, пусть $ucl(A) = \K$, но $A = T(A)$ не является канонической группой. По лемме ... \todo{Написать лемму} $A = A_{red} \oplus A_d$, где $A_d$ -- делимая $p$-группа, $A_{red}$ -- разложимая в прямую сумму циклических $p$-групп группа. В силу равенства $ucl(A) = \K$ следует, что $A_d \cong C^l(p^\infty)$ и пусть $A_{red} = C^{w_1}(p) \oplus \cdots \oplus C^{w_a}(p^a) \oplus T'$, где $T' = C^{w_{a+1}}(p^{a+1}) \oplus \cdots \oplus C^{w_b}(p^b)$, где $a, b$ и $l$ -- основные параметры универсального класса $\K$. Если $a > 1$ и хотя бы один из кардиналов $w_1, \ldots, w_{a-1}$ не равен нулю, то в $A$ существует элемент $a$ такой, что уравнение $px = a$ неразрешимо в $A$, то группа $\langle A, x\rangle \in \K$, и следовательно, $A \notin \Kec$. Итак мы показали, что если $A \in \Kec$, то $w_1 = \ldots = w_{a-1} = 0$ и $A_{red} = С^{w_a} (p^a) \oplus T'$\todo{Проверить степень и индекс}. Если группа $T$ неизоморфна группе $T'$, то $T'$ изоморфна подгруппе $T$. В этом случае существует конечная группа $C \in FG(\K)$, такая, что $C \notin FG(A)$, что противоречит условию $ucl(A) = \K$. Итак, доказано, что $A_{red} = C^{w_a}(p^a) \oplus T$ и, следовательно, группа $A \in C^*(\K)$. 

Докажем пункт 3 теоремы. Первая часть этого пункта доказывается теми же рассуждениями, что и в пункте~2 относительно группы $T(A)$. В нашем случае $A = T(A)$.

В подслучае $\delta(\K) = 1$, подгруппа $\Z$, которая содержится в $\K$ не дискриминируется классом групп $T(\K)$, поэтому, в данном случае необходимо, чтобы $B$ была ненулевой делимой группой без кручения. С другой стороны, если $A$ имеет вид из условия теоремы и конечное расширение $A(x_1, \ldots, x_n)$ принадлежит $\K$, то эта группа $A$-дискриминируется $A$. $\square$


Для произвольного главного класса $\K$ описание групп из класса $\Kec$ дается следующим образом.

\begin{theorem}\label{th:GroupInKecAny}
Пусть $\K$ -- произвольный главный класс, и пусть $\K_p = \K \cap \A_p$, где $p$ -- простые числа, и группа $A \in \Kec$. Тогда
\begin{enumerate}
\item $A = \bigoplus\limits_{p \in \P} T_p(A) \oplus B$, где $B$ -- делимая группа без кручения (возможно, что $B = 0$), а группа $T_p(A) \in C^*(\K_p)$, $p$ -- простое число;

\item Группа $B$ с необходимостью является ненулевой только в случае, если $T(\K)$ -- ограниченный, а $\K$ -- неограниченный класс;

\item $ucl(A) = \K$.
\end{enumerate}
\end{theorem}

\proof В силу леммы~\ref{lemma:AinOrnotinKec} достаточно доказать, что для любого конечного расширения $D = A(x_1, \ldots, x_n)$ вложение $A$ в $D$ является сервантным и $P(D) = P(A)$.

Пусть $A \in \Kec$. Тогда $A = \bigoplus\limits_{p \in \P} T_p(A) \oplus B$, где $B$ -- делимая группа без кручения. Тогда группа $\bigoplus\limits_{p \in \P} T_p(A) \oplus B$ должна принадлежать $\Kec_p$ и, следовательно, по ??? \todo{написано по предыдущему пункту, но его нет} $T_p(A) \in C^*(\K_p)$.

Наоборот. Пусть группа $A$ имеет вид из формулировки теоремы, и пусть $D = A(x_1, \ldots, x_n)$, $D = \bigoplus\limits_{p \in \P} T_p(D)$. Если вложение $A$ в $D$ не является сервантным, то существует $a \in A$, такой, что уравнение $nx = a$ не имеет решения в $A$, но имеет решение в $D$. Пусть $a = (\ldots, a_p, \ldots)$, тогда существует $a_p$ такое, что $n x_p = a_p$, но не имеет решения в $T_p(A)$. Это противоречит теореме~\ref{th:GroupsInKecAp}.

Итак, $D = A \oplus C \oplus \Z^k$, где $C$ -- конечная группа, и пусть $C = \bigoplus\limits_{p \in \{p_1, \ldots, p_l\}} T_p(C)$, где $T_{p_i}(C) \neq 0$. По условию $T_{p_i}(C) \in T_p(A_i)-\Dis(T_p(A_i))$, а потому $A \oplus C \in A-\Dis(A)$. Если $k \neq 0$, то $\delta(\K) = 1$, и следовательно, $A \oplus C \oplus \Z^k \in A-\Dis(A)$. 

Доказательство пункта 3 теоремы следует из того, что $A \in C^*(\K)$ для главного класса. $\square$


И наконец, описание групп из $\Kec$ для неглавного класса $\K$ содержится в следующем результате.

\begin{theorem}
Пусть $\K$ -- неглавный универсальный класс, и пусть $\K = \bigcup\limits_{i \in I} \K_i$, где $\K_i$ -- максимальный главный подкласс в $\K$, $i \in I$. Тогда:
\begin{enumerate}
\item $\Kec = \bigcup\limits_{i \in I} \Kec_i$, если $\delta(T(\K)) = 1$ или $\delta(\K) = 0$;

\item $\Kec = \bigcup\limits_{i \in I} \Kec_i \oplus B$, где $B$ -- ненулевая делимая группа без кручения, если $\delta(T(\K)) = 0$, но $\delta(\K) = 1$.
\end{enumerate}
\end{theorem}

\proof Пусть $A \in \Kec$. Докажем, что существует индекс $i \in I$ такой, что $A \in \K_i$. Ясно, что такой индекс $i$ существует в $I$, ибо $ucl(A)$ -- главный класс и ясно, по определению, что $A \in \Kec_i$. Допустим, что существует индекс $j \in I$ такой, что $A \in \Kec_j$. Тогда по теореме~\ref{th:GroupInKecAny} $ucl(A) = \K_i = \K_j$, что не так. $\square$


\begin{theorem}\label{th:AxiomsOfKec}
Пусть $\K$ -- главный универсальный класс групп. Тогда:
\begin{enumerate}
\item Класс $\Kec$ аксиоматизируем и, следовательно, $\Tha(\K)$ имеет модельный компаньон $\Th(\Kec)$;
\item Если $\delta(\K) = 0$ или $\delta(T(\K)) = 1$, то $\Th(\Kec)$ полна и совпадает с $\Th(C(\K))$, где $C(\K)$ -- каноническая группа класса $\K$. Если $\delta(\K) = 1$ и $\delta(T(\K)) = 0$, то $\Th(\Kec) = \Th(A)$, где $A = T(C(\K)) \oplus B$, где $B$ -- ненулевая делимая абелева группа без кручения.
\end{enumerate}
\end{theorem}

Отметим, что в пункте~1 теоремы~\ref{th:AxiomsOfKec} достаточно доказать только аксиоматизируемость класса $\Kec$, так как в силу теоремы Эклофа-Саббаха: теория $T$ имеет модельный компаньон, если и только если, $\Kec$ -- аксиоматизируемый класс, где $\K$ -- класс моделей теории $T$. Определения модельного компаньона для теории $T$ и доказательство теоремы Эклофа-Саббаха можно найти в \cite{Mac}.


\proof Разбором случав, будем доказывать сразу оба пункта теоремы.

Случай $\delta(T(\K)) = 1$. В этом случае, если $A \in \Kec$, то по теореме~\ref{th:GroupInKecAny} $A = T(A) \oplus B$, где $T(A) \in C^*(\K)$, $B$ -- делимая группа без кручения либо нулевая. По теореме Шмелевой \cite{Szm} $\Th(A) = \Th(T(A))$. Поэтому для доказательства теоремы достаточно доказать, что:
\begin{enumerate}
\item если $A'$ другая группа из $\Kec$, $A' = T(A') \oplus B'$, то $\Th(A') = \Th(T(A')) = \Th(T(A))$, где $T(A') \in C^*(\K)$ и $B'$ -- делимая группа без кручения;

\item если группа $A'' \equiv A$, то $A'' = T(A'') \oplus B''$, где $T(A'') \in C^*(\K)$, $B''$ -- делимая группа без кручения.
\end{enumerate}

Докажем эти утверждения в предположении, что $\K \subseteq \A_p$. Если $p = 0$, то эти утверждения очевидны. Пусть $p$ -- просто число и числа $a, b$ и $l$ -- параметры класса $\K$. Тогда
$$T(A) = C(p^a)^{w_1} \oplus T \oplus C^l(p^\infty),$$
$$T(A') = C(p^a)^{w_2} \oplus T \oplus C^l(p^\infty).$$
Если $a = 0$, $l \in \N$, то $T(A) \cong T(A')$, и если $A \equiv A''$, то по теореме Шмелевой, и далее по теореме ?? \todo{Проверить ссылку}, $A'' = T \oplus C^l(p^\infty) \oplus B''$, где $B''$ -- делимая группа без кручения, а потому $A'' \in \Kec$ и $\Th(A'') = \Th(A)$.

Если $a \neq 0$, $l \in \N$, то $w_1$ и $w_2$ -- бесконечные кардиналы, и по теореме Шмелевой $T(A) \equiv T(A')$, а потому $A \equiv A'$. Наоборот, пусть $A \equiv A''$, тогда $A'' \in \K = \K_p$ и $T_p(A) \equiv T_p(A'')$ и $T_q(A) = 0$ для всех $q \in \P$, $q \neq p$, и $\gamma_{p,k}(A) = \gamma_{p,k}(A'')$. Следовательно, $T_p(A'') = C^{w_3}(p^a) \oplus T \oplus C^l(p^\infty)$, где $w_3$ -- бесконечный кардинал. Отсюда следует, что $A'' \in \Kec$.

Если $l = \aleph_0$, то $a = b = 0$ и $\K_p = \A_p$. Тогда, если $A \in \Kec$, то $T(A) = C^w(p^\infty) \oplus B$, где $B$ -- делимая группа без кручения и $\Th(A) = \Th(C^w(p^\infty))$, где $w$ -- бесконечный кардинал. Если $A'' = A$, то $A'' = C^{w_2}(p^\infty) \oplus B$ и $A'' \in \Kec$.

Если теперь $\K$ -- произвольный главный универсальный класс и $A \in \Kec$, то по теореме~\ref{th:GroupInKecAny} $A = \bigoplus\limits_{p \in \P} T_p(A) \oplus B$, где $B$ -- делимая группа без кручения и $T_p(A) \in C^*(\K_p)$. Если $A'$ -- другая группа из $\Kec$ и $A' = \bigoplus\limits_{p \in \P} T_p(A') \oplus B'$, то для всех $p \in \P$ $T_p(A) \in \Kec_p$ и $T_p(A') \in \Kec_p$ и по предыдущим рассуждениям, $T_p(A) \equiv T_p(A')$ и, следовательно, $A \equiv A'$. Наоборот, если $A'' \equiv A$ и $T(A'') = \bigoplus\limits_{p \in \P} T_p(A'')$, то $T_p(A'') \equiv T_p(A)$ для всех простых чисел $p$. Так как $T_p(A) \in C^*(\K_p)$, то $T_p(A'') \in C^*(\K_p)$. Остается доказать, что $A'' \Big/ T(A'')$ -- делимая группа без кручения $B''$, и тогда $A'' = T(A'') \oplus B'' \in C^*(\K_p)$. Для этого, достаточно доказать $\alpha_{p,k}(B'') = 0$ для всех $p \in \P$, $k \in \N$. Если это не так, то $\alpha_{p,k}(B'') \neq 0$ для некоторых $p$ и $k$. Так как $A \equiv T(A) \equiv A''$, то $\alpha_{p,k}(T(A)) = \alpha_{p,k}(A'')$ равны символу $\infty$ для всех $k$. Если $l \in \N$, то это невозможно, что следует из канонического вида групп из $C^*(\K)$. Если $l = \aleph_0$ $C^*(\K) = C^w(p^\infty)$, где $w$ -- бесконечный ординал, то $\alpha_{p,k}(T_p(A)) = 0$. Следовательно, группа $B''$ является абелевой группой без кручения, и группа $A'' \in \Kec$.

Случай, когда $\delta(\K) = 0$ разбирается нетрудно. Если $A \in \Kec$, то $A = C^w(p^a) \oplus T$. Если $A' \in \Kec$, то $A' = C^{w'}(p^a) \oplus T$, где либо $w = w' = 0$, то $A = T$ и $A' = T$, либо $w, w'$ -- бесконечные кардиналы. Если $A'' \equiv A$, то $A'' = C^{w''}(p^a) \oplus T$ и $A'' \in \Kec$.

И наконец, случай когда $\delta(T(\K)) = 0$ и $\delta(\K) = 1$. Пусть $\K \subseteq \A_p$ и $A \in \Kec$, тогда $A = C^w(p^a) \oplus T \oplus B$, где $B \neq 0$ -- делимая группа без кручения. Если $A' \in \Kec$, то понятно, что $A \equiv A'$. Если $A'' \equiv A$, то $A'' = C^{w''}(p^a) \oplus T \oplus B''$ и $B'' \neq 0$ -- делимая группа без кручения.

Случай произвольного $\K$ с параметрами выше легко сводится к примарному случаю. 

Во всех случаях $\Kec$ -- аксиоматизируемый класс. По теореме~??? \todo{Вставить теорему для ссылки} \cite{Mac} $\Tha(\K)$ имеет модельный компаньон $\Th(\Kec)$. $\square$



\section{Структура $\K$-однородных абелевых групп}

\begin{theorem}
Для любого главного универсального класса $\K$ класс $\K$-однородных групп $\Kh$ содержит класс экзистенциально замкнутых групп $\Kec$.
\end{theorem}

\todo{Доказать теорему}



%
% Список литературы
%
\begin{thebibliography}{99}

\bibitem{DMR1} E.~Daniyarova, A.~Miasnikov, V.~Remeslennikov, \textit{Unification theorems in algebraic geometry}, // Algebra and Discrete Mathematics, 1 (2008), 80--111, arXiv: 0808.2522.

\bibitem{DMR2} Э.Ю.~Даниярова, А.Г.~Мясников, В.Н.~Ремесленников, \textit{Алгебраическая геометрия над алгебраическими системами. II. Основания}, // Фундамент. и прикл. матем., 17:1 (2012), 65--106, arXiv: 1002.3562.

\bibitem{Prest1} M.~Prest, \textit{Model Theory and Modules}, London Mathematical Society Lecture Notes Series Vol. 130, Cambridge University Press, Cambridge (1988), 400pp.

\bibitem{Prest2} M.~Prest, \textit{Model theory and modules}. // In: M Hazewinkel, (eds). Handbook of Algebra. Elsevier, (2003), p.~227--253. 

\bibitem{RR} В.Н.~Ремесленников, В.А.~Романьков, \textit{Теоретико-модельные и алгоритмические вопросы теории групп}, // Итоги науки и техн. Сер. Алгебра. Топол. Геом., 21, ВИНИТИ, М., (1983), c.~3-79. 

\bibitem{Hisamiev1} N.G.~Hisamiev, \textit{Constructive abelian groups}, // Handbook of recursive Mathematic, V 2, (1998), p.~1177--1232.

\bibitem{Palutin1} Е.А.~Палютин, \textit{$P$-спектры абелевых групп}, // Алгебра и логика, 53:2 (2014), с.~216--255.

\bibitem{Palutin2} Е.А.~Палютин, \textit{$P$-стабильные абелевы группы}, // Алгебра и логика, 52:5 (2013), с.~606--631.

\bibitem{Szm} W.~Szmielew, \textit{Elementary properties of Abelian groups}, // Fundamenta Mathematica, 41, (1955), p.~203--271.

\bibitem{Fuchs1} Л.~Фукс, \textit{Бесконечные абелевы группы}. Том~1. М.: Мир, 1974.~-- 336 с.

\bibitem{Fuchs2} Л.~Фукс, \textit{Бесконечные абелевы группы}. Том~2. М.: Мир, 1977.~-- 415 с.

\bibitem{Ershov} Ю.Л.~Ершов, \textit{Проблемы разрешимости и конструктивные модели.}~--- М.: Наука, 1980.

\bibitem{Hodges} W.~Hodges, \textit{Model Theory},~--- Cambridge University Press, 1993. 

\bibitem{Mac} A.~Macintyre, \textit{Model Completeness}, // Handbook Of Mathematical Logic, ed. by J.~Barwise, North Holland, Amsterdam (1977), pp.139--180.

\bibitem{Z} M.~Ziegler, \textit{Model theory and modules}, // Ann. Pure Applied Logic, 26 (1984), p.~149--213.

\end{thebibliography}


% \bibliographystyle{abbrv}
% \bibliography{bibfile}
\end{document}